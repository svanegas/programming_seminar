\documentclass{beamer}
\usepackage[utf8]{inputenc}
\usepackage[spanish]{babel}
\usepackage{hyperref}
\usepackage{verbatim}
\usepackage{listings}

\setbeamercovered{invisible}
\usetheme{Frankfurt}
\usefonttheme{serif}

% Configurar los listings (Códigos)
\renewcommand{\lstlistingname}{Código}
\lstset{
	language=C++,               % Lenguaje
	basicstyle=\ttfamily\tiny,  % Tipo de fuente
	keywordstyle=\color{blue},  % Color de palabras clave
	stringstyle=\color{red},    % Color de strings
	commentstyle=\color{gray},  % Color de comentarios
	showstringspaces=false,     % No muestrar el _ cuando el string tiene espacios
	breaklines = true,          % Partir las líneas largas
	breakatwhitespace=true,	    % Partir las líneas en un espacio
	numbers=left,				% Numerar las líneas a la izq
	numberstyle=\tiny,			% Poner los números de las líneas pequeños
	numberblanklines=true,      % Numerar las líneas en blanco
	columns=fullflexible,       % No perder el formato al dejar los espacios
	keepspaces=true,   			% Dejar los espacios insertados
	frame=tb,					% Poner el recuadro
}

\AtBeginSection[]{%
  \begin{frame}<beamer>
    \frametitle{Contenido}
    \tableofcontents[sectionstyle=show/hide,subsectionstyle=hide/show/hide]
  \end{frame}
  \addtocounter{framenumber}{-1}% If you don't want them to affect the slide number
}

\title{Semillero de Programación}
\subtitle{Cap 2. Estructuras de datos y bibliotecas}
\author{Santiago Vanegas Gil}

\institute{Universidad EAFIT}
\date{8 de agosto del 2014}

\begin{document}

\begin{frame}
	\titlepage
\end{frame}

\begin{frame}
	\frametitle{Contenido}
	\tableofcontents
\end{frame}


\begin{section}{Estructuras de datos lineales}
\begin{subsection}{Arreglos estáticos}
	\begin{frame}[fragile]
		\frametitle{Arreglos estáticos}
		\begin{itemize}
			\item{Es una colección de datos secuenciales que son guardados para luego ser consultados basandose en sus índices.}
			\item{Es claramente la estructura más común en maratones de programación.}
			\item{Típicamente se usan arreglos de 1, 2 y 3 dimensiones como máximo.}
			\item{Por ejemplo, un arreglo de 10 posiciones llamado $a$ puede ser representado así:}
			\begin{figure}
				\includegraphics[width = 0.7\textwidth]{src/arreglo.jpg}
			\end{figure}			
		\end{itemize}
	\end{frame}
	
	\begin{frame}[fragile]
		\frametitle{Arreglos estáticos}
		A continuación, aprenderemos cómo declarar arreglos estáticos en C++.
		\begin{block}{Declaración de arreglos estáticos en C++}
			\begin{verbatim}
				tipo_de_dato nombre [número_de_elementos];
			\end{verbatim}
			Ejemplos:\\
			\begin{verbatim}
				int arr [10];
				string words [50];
			\end{verbatim}			
		\end{block}		
	\end{frame}
	
	\begin{frame}[fragile]
		\frametitle{Arreglos estáticos}
		Ahora veamos cómo se declaran en Java.
		\begin{block}{Declaración de arreglos estáticos en Java}
			\begin{verbatim}
				tipo_de_dato [] nombre = new tipo_de_dato [tamaño];
			\end{verbatim}
			Ejemplo:
			\begin{verbatim}
				int [] nums = new int[10];
				String [] words = new String[5];
			\end{verbatim}	
		\end{block}
		\begin{figure}
		\end{figure}		
	\end{frame}
	
	\begin{frame}[fragile]
		\frametitle{Arreglos estáticos}
		\small{Un ejemplo de uso de arreglos estátios es leerlo y luego recorrerlo hasta encontrar un 0.}
		\begin{lstlisting}
			#include <iostream>
			
			using namespace std;

			const int MAXN = 10;
			int nums[MAXN];
			
			int
			main() {
			    for (int i = 0; i < sizeof(nums) / sizeof(nums[0]); i++) {
			        cin >> nums[i];
			    }
			    for (int i = 0; i < sizeof(nums) / sizeof(nums[0]); i++) {
			        if (nums[i] == 0) {
			            cout << "I found a 0 in position: " << i << endl;
			            break;
			        }
			    }
			    return 0;
			}
		\end{lstlisting}		
	\end{frame}	
\end{subsection}
\begin{subsection}{Arreglos dinámicos}
	\begin{frame}[fragile]
		\frametitle{Arreglos dinámicos}
		\begin{itemize}
			\item {Los arreglos dinámicos o vectores son estructuras similares a los arreglos estáticos, excepto que son diseñados para permitir cambio de tamaño en tiempo de ejecución.}
			\item {Al igual que los arreglos, los elementos pueden ser accedidos por medio del índice de su posición, empezando por en índice 0.}
			\item {Las operaciones comúnes en vectores en C++ son: $push\_back()$, operador $[]$ o $at()$, $clear()$, $erase()$, $size()$.}
			\item {En C++, para hacer uso de vectores es necesario importar la biblioteca $vector$, y en Java $ArrayList$.}
		\end{itemize}
		\begin{figure}
			\includegraphics[width = 0.7\textwidth]{src/vector.jpg}
		\end{figure}
	\end{frame}
	
	\begin{frame}[fragile]
		\frametitle{Arreglos dinámicos}
		Veamos un ejemplo del uso de vectores en C++:
		\begin{lstlisting}
			#include <iostream>
			#include <vector> //Remember to import the library

			using namespace std;
			
			int
			main() {
			    vector <int> nums(5, 5); //Initial size 5 with values {5, 5, 5, 5, 5}
			    for (int i = 0; i < nums.size(); i++) {
			        nums[i] = i * 2; //Store new value in position i
			    }
			    for (int i = 0; i < nums.size(); i++) {
			        cout << nums[i] << " ";
			    }
			    //Output will be: 0 2 4 6 8
			    return 0;
			}
		\end{lstlisting}
	\end{frame}
\end{subsection}
\begin{subsection}{Arreglo de booleanos (bitset)}
	\begin{frame}[fragile]
		\frametitle{Arreglo de booleanos}
		\begin{itemize}
			\item {Los arreglos de booleanos o $bitsets$ sólo se almacenan valores boleanos, esto es, $1/true$ o $0/false$.}
			\item {Permite almacenar muchos bits de una forma eficiente}
			\item {La biblioteca $bitset$ en C++ soporta operaciones útiles como $reset()$, $set()$, $test()$ y el operador [].}
		\end{itemize}
	\end{frame}
	
	\begin{frame}[fragile]
		\frametitle{Arreglo de booleanos}
		Veamos un ejemplo del uso de bitsets en C++:
		\begin{lstlisting}
			#include <iostream>
			#include <bitset> //Remember to import the library
			#include <string> //Needed below

			using namespace std;
			
			int
			main() {
			    bitset<16> bits1; //Create an empty bitset
			    bitset<16> bits2 (0xFA2); //Create a bitset from an hexadecimal value
			    bitset<16> bits3 (string("1110101")); //Bitset from a string
			    
			    cout << "bits1: " << bits1 << endl; //bits1: 0000000000000000
			    cout << "bits2: " << bits2 << endl; //bits2: 0000111110100010
			    cout << "bits3: " << bits3 << endl; //bits3: 0000000001110101
			    cout << "bits2 has " << bits2.count() << " ones" << endl; //bits2 has 7 ones
			    cout << "bits2 is: " << bits2.to_ulong() << endl; //bits2 is: 4002
			    return 0;
			}
		\end{lstlisting}
	\end{frame}
\end{subsection}

%% MÁSCARAS DE BITS %%
\begin{subsection}{Máscaras de bits}
	\begin{frame}[fragile]
		\frametitle{Máscaras de bits}
		\begin{itemize}
			\item {Podemos usar enteros para representar pequeños conjutos de valores booleanos.}
			\item {Usar operaciones de bit a bit con enteros, en algunos casos, resulta ser mucho más eficiente comparado con vectores de booleanos o $bitsets$.}
			\item {Pueden facilitar bastante la implementación de un problema y ganar tiempo en una maratón.}
		\end{itemize}
	\end{frame}

	\begin{frame}[fragile]
		\frametitle{Máscaras de bits}
		Antes de comenzar con todas las operaciones, recuerda que los bits se cuentan de derecha a izquierda, siendo la derecha el bit menos significativo y la izquierda el bit más significativo.
		\begin{block}{Operación shifting}
			A un conjunto de bits se le puede hacer $Shifting$ hacia la izquierda y derecha, cada dirección se representa con el símbolo $<<$ y $>>$ respectivamente.\\
		Ejemplo:\\
		Si se tiene el siguiente conjunto de bits y se le hace $Shift$ a la derecha 3 veces obtendremos el siguiente resultado:\\
			\begin{itemize}
				\item{0010010} - Conjunto original
				\item{0001001} - Primer shift
				\item{0000100} - Segundo shift (Pérdida bit menos significativo)
				\item{0000010} - Tercer shift
			\end{itemize}
		\end{block}		
	\end{frame}

	\begin{frame}[fragile]
		\frametitle{Máscaras de bits}
		\begin{block}{Cómo saber si un bit está prendido}
			Sea S el conjunto de bits, podemos saberlo con la siguiente instrucción:
			\begin{verbatim}
				S & (1 << j)
			\end{verbatim}
			Donde j es la posición del bit que se quiere consultar.\\
			Ilustración:
			\begin{verbatim}
				Con j  = 4
				S      = 101010 (bin)
				1 << 4 = 010000 (bin)
					         ------ AND
					         000000 <- El resultado es 0
					                   el bit no está prendido.
			\end{verbatim}		
		\end{block}		
	\end{frame}

	\begin{frame}[fragile]
		\frametitle{Máscaras de bits}
		\begin{block}{Encender el bit j (base 0) de un set}
			Se debe usar la operación bit a bit \textbf{OR}, veamos:
			\begin{verbatim}
				S |= (1 << j)
			\end{verbatim}
			\begin{lstlisting}
				S = 34 (base 10) = 100010 (base 2)
				j = 3, 1 << j    = 001000 <- bit 1 desplazado 3 veces
                  				------- OR
				S = 42 (base 10) = 101010 (base 2)	
			\end{lstlisting}
		\end{block}		
	\end{frame}
	
	\begin{frame}[fragile]
		\frametitle{Máscaras de bits}
		\begin{block}{Apagar el bit j (base 0) de un set}
			Se debe usar la operación bit a bit \textbf{AND}, veamos:
			\begin{verbatim}
				S &= ~(1 << j)
			\end{verbatim}
			\begin{lstlisting}
				S = 42 (base 10) = 101010 (base 2)
				j = 1, ~(1 << j) = 111101 <-- Shift y NOT
                  						------- AND
				S = 40 (base 10) = 101000	
			\end{lstlisting}	
		\end{block}		
	\end{frame}
	
	\begin{frame}[fragile]
		\frametitle{Máscaras de bits}
		\begin{block}{Multiplicar o dividir por potencias de 2}
			Sólo se necesita hacer $Shift$ a los bits del entero la cantidad de veces que equivalen a la potencia.
			\begin{lstlisting}
				S                  = 34 (base 10) =  100010 (base 2)
				S = S << 1 = S * 2 = 68 (base 10) = 1000100 (base 2)
				S = S >> 2 = S / 4 = 17 (base 10) =   10001 (base 2)
				S = S >> 1 = S / 2 =  8 (base 10) =    1000 (base 2)
			\end{lstlisting}	
			Notar que la división es entera, por lo que se corre el riesgo de perder bits menos significativos.	
		\end{block}		
	\end{frame}
	
\end{subsection}
\begin{subsection}{Pilas}

	\begin{frame}[fragile]
		\frametitle{Pilas}
		\begin{itemize}
			\item {Sólo permite una operación aplicada al tope de la pila, es decir, meter o sacar el último elemento del contenedor.}
			\item {\textbf{El úlltimo elemento que ingresó es el primer elemento en salir (LIFO)}}
		\end{itemize}
		\begin{figure}
			\includegraphics[width = 0.7\textwidth]{src/pila.jpg}
		\end{figure}
	\end{frame}

	\begin{frame}[fragile]
		\frametitle{Pilas}
		Las operaciones que se pueden usar en una pila son:
		\begin{block}{Operaciones soportadas por una pila}
			\textbf{push(x)} - Inserta el elemento x al final de la pila.\\
			\textbf{pop()} - Remueve el último elemento de la pila.\\
			\textbf{top()} - Retorna el último elemento de la pila, sin removerlo.\\
			\textbf{empty()} - Retorna verdadero si la pila está vacía.\\
			\textbf{size()} - Retorna el tamaño de la pila.			
		\end{block}		
	\end{frame}
	
	\begin{frame}[fragile]
		\frametitle{Pilas}
		Veamos un ejemplo de implementación de pila usando la librería de C++ $stack$.
		\begin{lstlisting}
			#include <iostream>
			#include <stack>            // Remember to include stack
			using namespace std;

			int main(){
			   stack <int> s;            // Create an integer stack
			   s.push(10);               // Insert 10
			   s.push(-1);               // Insert -1
			   cout << s.top() << endl;  // Print -1
			   s.pop();                  // Remove -1
			   cout << s.top() << endl;  // Print 10
			   cout << s.size() << endl; // The stack size is 1
			   return 0;
			}
		\end{lstlisting}
	\end{frame}

\end{subsection}

\begin{subsection}{Colas}
	\begin{frame}[fragile]
		\frametitle{Colas}
		\begin{itemize}
			\item {Las operaciones básicas son insertar al final de la cola, y eliminar del frente de la cola.}
			\item {\textbf{El primer elemento insertado es el primero en salir de la cola (FIFO)}}
		\end{itemize}
		\begin{figure}
			\includegraphics[width = 0.7\textwidth]{src/cola.jpg}
		\end{figure}
	\end{frame}

	\begin{frame}[fragile]
		\frametitle{Colas}
		Las operaciones que se pueden usar en una cola son:
		\begin{block}{Operaciones soportadas por una pila}
			\textbf{push(x)} - Inserta el elemento x al final de la cola.\\
			\textbf{pop()} - Remueve el primer elemento (frontal) de la cola.\\
			\textbf{front()} - Retorna el primer elemento de la cola, sin removerlo.\\
			\textbf{empty()} - Retorna verdadero si la cola está vacía.\\
			\textbf{size()} - Retorna el tamaño de la cola.			
		\end{block}		
	\end{frame}
	
	\begin{frame}[fragile]
		\frametitle{Colas}
		Veamos un ejemplo de implementación de cola usando la librería de C++ $queue$.
		\begin{lstlisting}
			#include <iostream>
			#include <queue>               // Remember to include queue
			using namespace std;

			int main(){
			   queue <int> q;              // Create an integer queue
			   q.push(10);                 // Insert 10
			   q.push(-1);                 // Insert -1
			   cout << q.front() << endl;  // Print 10
			   q.pop();                    // Delete 10
			   cout << q.front() << endl;  // Print -1
			   cout << q.size() << endl;   // The queue size is 1
			   return 0;
			}
		\end{lstlisting}
	\end{frame}

\end{subsection}
		
\end{section}

\begin{section}{Estructuras de datos no lineales}
	\begin{frame}
		\frametitle{Estructuras de datos no lineales}
		\begin{itemize}
			\item {Para algunos problemas, el almacenamiento lineal no es la mejor opción por su eficiencia en ciertas operaciones.}
			\item {Colecciones dinámicas no lineales como $Maps$ o $Hash Tables$ son las más adecuadas debido a su rendimiento en operaciones de inserción, búsqueda y eliminación.}
		\end{itemize}
	\end{frame}

\begin{subsection}{Arbol binario de búsqueda}
	\begin{frame}
		\frametitle{Arbol binario de búsqueda}
		\begin{itemize}
			\item {Es una forma de organizar los datos en una estructura de árbol.}
			\item {Los items en una rama izquierda de un elemento x son menores que x, y los items en una rama derecha de x son mayores o iguales a x.}
			\item {En C++, implementaciones de árboles binarios de búsqueda están las bibliotecas $map$ y $set$}
		\end{itemize}
		\begin{figure}
			\includegraphics[scale=0.6]{src/bst.jpg}
		\end{figure}
	\end{frame}

%%MAPA%%
\begin{subsubsection}{Mapa}
	\begin{frame}[fragile]
		\frametitle{Mapa}
		\begin{block}{Mapa}
			Un mapa es un contenedor que guarda parejas de elementos. El primer elemento de la pareja (key) sirve para identificarla y el segundo elemento (mapped value) es el valor asociado a la llave.\\
		\end{block}
		\begin{block}{Declaración}
			\verb|#include <map>|\\
			\verb|map <tipo_dato_key, tipo_dato_value> nombre;|\\
			Ejemplos:\\
			\verb|map <string, int> m;|\\
			\verb|map <char, int> char2int;|
		\end{block}
	\end{frame}
	
	\begin{frame}[fragile]
		\frametitle{Acceso a elementos de un mapa}
		Los elementos de un mapa se llaman por su llave así:\\ \quad \\
		\verb|map <string, int> m;|\\
		\verb|m["Hola"] = 3;|\\
		\verb|int a = m["Cangrejo"];|\\ \quad \\
		
		Para ingresar un elemento se puede hacer así:\\
		\verb|if (m.count["Nuevo"] == 0) m["Nuevo"] = 123;|\\ \quad \\
		Con la función \verb|count| se busca cuántas veces está el elemento con la llave \verb|"Nuevo"|. Si el elemento no está, cuando accedemos a él con \verb|[]| éste se crea automáticamente.\\
		Cuando accedo con \verb|[]| a un elemento del mapa que no existe este se crea con valores por defecto. Para strings es el string vacío \verb|""| y para enteros es el número 0.
	\end{frame}
	
	\begin{frame}
		\frametitle{Complejidad del mapa}
		\begin{block}{Complejidad}
			\begin{itemize}
				\item Insertar / acceder un elemento al mapa es $O(log\,n \times k)$ donde $n$ es el número de elementos en el mapa y $k$ es el tiempo que toma comparar dos llaves del mapa.
				\item Comparar dos enteros es $O(1)$, comparar dos strings es $O(m)$ donde $m$ es la longitud de los strings.
				\item Insertar / acceder un elemento a un mapa con llaves strings es $O(log\,n \times m)$ donde $n$ son los elementos del mapa y $m$ es la longitud del string.
			\end{itemize}
		\end{block}
		\begin{block}{Nota}
			Los elementos de un mapa se almacenan en orden de acuerdo a una función de comparación. Por defecto la función de comparación es la de menor, es decir que los elementos se almacenan de menor a mayor.
		\end{block}
	\end{frame}
	
	\begin{frame}[fragile]
		\frametitle{Recorrer un mapa}
		Para recorrer un mapa es necesario usar iteradores.\\
		\begin{lstlisting}
		#include <iostream>
		#include <map>
		using namespace std;

		int main(){
		   map <string, int> m;
		   m["b"] = 4;
		   m["bc"] = 1;
		   m["a"] = 3;
		   map <string, int> :: iterator it;
		   for (it = m.begin(); it != m.end(); it++){
		      cout << "( " << it->first << " " << it->second << " ) ";
		      // cout << "( " << (*it).first << " " << (*it).second << " ) ";
		   }
		   return 0;
		}
		// La funcion imprime ( a 3 ) ( b 4 ) ( bc 1 )
		\end{lstlisting}
	\end{frame}
	
	\begin{frame}[fragile]
		\frametitle{Otras funciones en el mapa}
		Otras funciones que se pueden hacer con el mapa son:\\
		\begin{itemize}
			\item Recorrerlo al revés con \verb|rbegin| y \verb|rend|
			\item Obtener el tamaño con \verb|size|
			\item Borrar el contenido con \verb|clear|
			\item Insertar elementos (si no están antes) con \verb|emplace|
			\item Borrar elementos con \verb|erase|
			\item Buscar un elemento con \verb|find|
		\end{itemize}
		\quad \\
		Para más información mirar \url{http://www.cplusplus.com/reference/map/map/}
	\end{frame}
\end{subsubsection}
%% SET %%
\begin{subsubsection}{Set}
	\begin{frame}[fragile]
		\frametitle{Set}
		\begin{block}{Set}
			Un set es un contenedor que guarda conjuntos, es decir grupos de elementos iguales donde cada elemento aparece una sola vez.\\
			Los elementos de un set se almacenan en orden de acuerdo a una función de comparación. Por defecto la función de comparación es la de menor, es decir que los elementos se almacenan de menor a mayor.\\
		\end{block}
		\begin{block}{Declaración}
			\verb|#include <set>|\\
			\verb|set <tipo_dato> nombre;|\\
			Ejemplos:\\
			\verb|set <string> s;|\\
			\verb|set <int> amigos;|
		\end{block}
	\end{frame}

	\begin{frame}[fragile]
		\frametitle{Set}
		Sobre un set se pueden hacer las siguientes operaciones.
		\begin{description}
			\item[insert] Inserta un elemento al set. Ejemplo: \verb|amigos.insert(9)|
			\item[count] Cuenta cuántas veces aparece un elemento (0 o 1 vez). Ejemplo: \verb|amigos.count(3)|
			\item[find] Retorna un iterador al lugar donde está el elemento. Ejemplo: \verb|amigos.find(3)|
			\item[erase] Elimina un elemento del set. Ejemplo: \verb|amigos.erase(amigos.find(3))|
		\end{description}
		Todas las operaciones anteriores tienen una complejidad $O(log\,n \times k)$ donde $n$ es el número de elementos del set y $k$ es el tiempo que toma comparar dos elementos.\\
		Para más información mirar \url{http://www.cplusplus.com/reference/set/set/}
	\end{frame}

	\begin{frame}[fragile]
		\frametitle{Recorrer un set}
		Para recorrer un set es necesario usar iteradores.\\
		\begin{lstlisting}
		#include <iostream>
		#include <set>
		using namespace std;

		int main(){
		   set <int> s;
		   s.insert(4); 
		   s.insert(-1);
		   s.insert(3);
		   s.insert(4);
		   set <int> :: iterator it;
		   for (it = s.begin(); it != s.end(); it++){
		      cout << *it << " ";
		   }
		   return 0;
		}
		// La funcion imprime -1 3 4
		\end{lstlisting}
	\end{frame}
\end{subsubsection}
%% PRIORITY QUEUE %%
\begin{subsubsection}{Cola de prioridad}
	\begin{frame}[fragile]
		\frametitle{Cola de prioridad}
		Un heap se puede representar en C++ como una cola de prioridades así:
		\verb|#include <queue>|\\
		\verb|priority_queue <tipo_dato> nombre;|\\
		Ejemplos:\\
		\verb|priority_queue <int> heap;|\\
		\verb|priority_queue <pair <int, int> > q;|\\
		\quad \\
		Para más información mirar: \url{http://www.cplusplus.com/reference/queue/priority_queue/}
	\end{frame}
	
	
	\begin{frame}
		\frametitle{Operaciones}
		La cola de prioridades (heap) soporta las siguientes operaciones
		\begin{description}
			\item [push] Inserta un elemento
			\item [pop] Extrae un elemento
			\item [top] Retorna el máximo elemento de la cola (heap)
			\item [size] Retorna el tamaño de la cola (heap)
		\end{description}
		\quad \\
		La cola de prioridades se ordena de acuerdo a la función de ordenamiento $<$ (menor que) por lo que retorna el elemento con el cual todos los demás comparan menor que él, es decir, el mayor elemento.
	\end{frame}

	\begin{frame}
		\frametitle{GRACIAS}
		Esta presentación fue basada en el libro Competitive Programming 3 de Steven Halim y Felix Halim.
		\newline
		Ciertos elementos contenidos en las diapositivas fueron tomados de:
		\url{https://github.com/anaechavarria/SemilleroProgramacion/tree/master/Diapositivas}
	\end{frame}
\end{subsubsection}
\end{subsection}
\end{section}
\end{document}